\documentclass{article}
\usepackage[utf8]{inputenc}
\usepackage[french]{babel}
\usepackage{amsmath}
\usepackage{listings}
\usepackage{hyperref}
\usepackage{xcolor}
\usepackage{graphicx}
\usepackage[style=authoryear]{biblatex}
\usepackage{fancyhdr}
\pagestyle{fancy}

\fancyhf{}
\fancyhead[L]{Plan du séminaire : L'algorithme quantique}
\fancyhead[R]{RAETH et SANNA, L3STI}
\fancyfoot[C]{Page n°\thepage}

\addbibresource{ref.bib}

\title{Plan du séminaire :\\L'algorithme quantique}
\author{RAETH Léandre, SANNA Thomas\\L3 Sciences et Technologie  P. Informatique}
\date{\today}

\begin{document}

\maketitle

\section{Introduction}
\subsection{Présentation du sujet}
\begin{itemize}
    \item L'informatique quantique est bien plus qu'un simple ordinateur "sous stéroïde" (\cite{V2F_2024}).
    \item Objectif du séminaire : comprendre pourquoi et comment les algorithmes quantiques surpassent les algorithmes classiques.
\end{itemize}
\subsection{Exemple introductif}
\begin{itemize}
    \item Recherche d'un élément dans une liste classique (\(\epsilon(n)\) en informatique classique vs. \(\epsilon(\sqrt{n})\) en quantique avec Grover).
\end{itemize}

\section{Informatique classique vs. Informatique quantique}
\subsection{Bits classiques}
\begin{itemize}
    \item Représentation en 0 et 1 via des transistors et du courant électrique.
    \item Opérations logiques via des portes (AND, OR, NOT...).
\end{itemize}
\subsection{Qubits quantiques}
\begin{itemize}
    \item Différents supports physiques (ions, supraconducteurs, spin d'électrons).
    \item Principales différences : Superposition, Intrication, Interférence.
\end{itemize}

\section{Propriétés fondamentales de l'informatique quantique}
\subsection{Superposition}
\begin{itemize}
    \item Un qubit peut être simultanément en état 0 et 1.
    \item Notation mathématique : \(|\psi\rangle = \alpha|0\rangle + \beta|1\rangle\). (\cite{youtubeYouTube})
\end{itemize}
\subsection{Intrication}
\begin{itemize}
    \item Deux qubits intriqués partagent un lien instantané, peu importe la distance.
\end{itemize}
\subsection{Interférence}
\begin{itemize}
    \item Permet de manipuler les probabilités des états mesurés. (\cite{dami})
\end{itemize}
\section{L'algorithme de Grover}
\subsection{Problématique}
\begin{itemize}
    \item Recherche d'un élément dans une liste non triée.
\end{itemize}
\subsection{Comparaison des algorithmes}
\begin{itemize}
    \item Algorithme classique : \(n\) étapes dans le pire des cas.
    \item Algorithme quantique : \(\sqrt{n}\) étapes grâce à Grover. (\cite{wikipediaAlgorithmeGrover})
\end{itemize}
\subsection{Fonctionnement en deux étapes}
\begin{enumerate}
    \item \textbf{Oracle} : Marquage de l'élément recherché.
    \item \textbf{Diffusion} : Amplification de la probabilité de la bonne réponse. (\cite{V2F_2024})
\end{enumerate}
\begin{itemize}
    \item Reprise de la boucle \(\sqrt{n}\) fois pour obtenir une probabilité optimale.
\end{itemize}

\section{Expérimentation : coder un algorithme quantique}
\subsection{Exécution d'un programme quantique}
\begin{itemize}
    \item Utilisation d’un simulateur ou d’un véritable ordinateur quantique via IBM Q Experience ou Microsoft Azure Quantum. (\cite{microsoftAzureQuantum}) (\cite{mediumGettingStarted})
\end{itemize}
\subsection{Exemple de code en Qiskit}
\begin{itemize}
    \item Exemple de code simple en Qiskit : exécution d'un circuit de Grover.
\end{itemize}

\section{Limites et perspectives}
\subsection{Défis actuels}
\begin{itemize}
    \item Fragilité des qubits (décohérence, erreurs de mesure).
    \item Nombre limité de qubits utilisables aujourd'hui. (\cite{V2F_2024})
\end{itemize}
\subsection{Futur de l'informatique quantique}
\begin{itemize}
    \item Augmentation du nombre de qubits et correction d'erreurs. (\cite{uopeopleWhatQuantum})
    \item Applications potentielles (cryptographie, optimisation, intelligence artificielle).
\end{itemize}

\section{Conclusion et Q\&A}
\subsection{Récapitulatif des points clés}
\begin{itemize}
    \item Importance de l'informatique quantique dans l'avenir. (\cite{dami})
\end{itemize}
\subsection{Session de questions-réponses}
\begin{itemize}
    \item Session de questions-réponses avec le public.
\end{itemize}

\break\printbibliography

\end{document}